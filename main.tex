\documentclass[UTF8,a4paper]{ctexart}
\usepackage{geometry}
\geometry{left=3cm,right=3cm,top=2.5cm,bottom=2.5cm}
\usepackage{titlesec}
\usepackage{titletoc}
\usepackage{setspace}
\usepackage{enumitem}
\usepackage[backend=biber,style=gb7714-2015]{biblatex}
\usepackage{fancyhdr}
\usepackage{hyperref}

% 全局格式设置
\setmainfont{Times New Roman}
\setCJKmainfont{SimSun}
\linespread{1.5}
\setlength{\parindent}{2em}

\titleformat{\section}{\centering\heiti\bfseries\zihao{-2}}{第\arabic{section}章}{1em}{}
\titleformat{\subsection}{\heiti\bfseries\zihao{4}}{\thesubsection}{1em}{}
\titleformat{\subsubsection}{\heiti\bfseries\zihao{-4}}{\thesubsubsection}{1em}{}
\ctexset{
    section = {
        name = {第,章},
        number = \arabic{section}
    }
}
% 目录格式定制
\dottedcontents{section}[0em]{\heiti\zihao{4}}{3.5em}{1pc}
\dottedcontents{subsection}[2em]{\heiti\zihao{5}}{2em}{1pc}

% 页眉页脚设置
\pagestyle{fancy}
% \fancyhf{}
\renewcommand{\headrulewidth}{0pt}

\begin{document}

\thispagestyle{empty}
% 封面页
\begin{center}
    \vspace*{2cm}
    {\heiti\bfseries\zihao{2} 中国引入QFII后的风险实证分析}
    
    \vspace{3cm}
    {\songti\bfseries\zihao{3}
    \begin{minipage}{0.4\textwidth}
    % 每行前方加 \hspace{1em} 可以继续调整对齐
    学位类型:同等学力 \\
    论文作者:赵某某 \\
    培养单位:统计学院 \\
    专业名称:统计学1班 \\
    指导教师:李某某\\
    \end{minipage}
    }
    
    \vfill
    {\heiti\bfseries\zihao{3} 二〇二五年二月}
\end{center}

\newpage
\thispagestyle{empty}
% 声明页
\section*{学位论文原创性声明}
\begin{spacing}{1.5}
{\songti\zihao{-4}
本人郑重声明:所呈交的学位论文,是本人在导师的指导下,
独立进行研究工作所取得的成果。除文中已经注明引用的内容外,
本论文不含任何其他个人或集体已经发表或撰写过的作品成果。
对本文所涉及的研究工作做出重要贡献的个人和集体,
均已在文中以明确方式标明。本人完全意识到本声明的法律责任由本人承担。

特此声明\\[2cm]

学位论文作者签名:\quad \quad \quad \quad 年 \quad 月 \quad 日}
\end{spacing}

\newpage
\thispagestyle{empty}
% 声明页
\section*{学位论文版权使用授权书}
\begin{spacing}{1.5}
{\songti\zihao{-4}
本人完全了解对外经济贸易大学关于收集、保存、使用学位论文的规定,
同意如下各项内容:
按照学校要求提交学位论文的印刷本和电子版本;
学校有权保存学位论文的印刷本和电子版,
并采用影印、缩印、扫描、数字化或其它手段保存论文;
学校有权提供目录检索以及提供本学位论文全文或部分的阅览服务;
学校有权按照有关规定向国家有关部门或者机构送交论文; 
学校可以采用影印、缩印或者其它方式合理使用学位论文,
或将学位论文的内容编入相关数据库供检索;
保密的学位论文在解密后遵守此规定。\\[2cm]

学位论文作者签名:\quad \quad \quad 年 \quad 月 \quad 日\\

导师签名:\quad \quad \quad \quad \quad \quad \quad 年 \quad 月 \quad 日}
\end{spacing}

\newpage
\clearpage
\pagenumbering{Roman} % 切换为阿拉伯数字
\setcounter{page}{1}   % 重置页码计数器
% 摘要页
\section*{摘\quad 要}
\begin{spacing}{1.5}
{\songti\zihao{-4}
在我国加入WTO之后......(摘要正文)\\[1em]

\noindent{\heiti\zihao{-4} 关键词:} QFII;QGARCH模型;风险}
\end{spacing}

\newpage
\section*{Abstract}
\begin{spacing}{1.5}
{\rmfamily\zihao{-4}
Hello......(英文摘要正文)\\[1em]

\noindent{\sffamily\zihao{-4} KEY WORDS:} Assets; Words; It}
\end{spacing}

% 目录页
\newpage
\tableofcontents

% 正文部分
\newpage
\clearpage
\pagenumbering{arabic} % 切换为阿拉伯数字
\setcounter{page}{1}   % 重置页码计数器
\section{引言}
\subsection{概述}
\subsubsection{背景分析}
1973年,美国经济学家麦金农(Ronald I.Mckinnon)......

% 参考文献
\newpage
\begin{thebibliography}{99}
\addcontentsline{toc}{section}{参考文献}
\bibitem{ref1} 李季, 王宇. 机构投资者:新金融景观[M]. 东北财经大学出版社, 2002.
\bibitem{ref2} 张丹. QFII实施现状及对我国证券市场的影响[J]. 商业研究, 2006(2):25-27.
\bibitem{ref3} John D. On Demand[J]. American Economic Review, 1956, 9:15-25.
\bibitem{ref4} John D. On Demand[M]. Oxford Press, 1956.
\end{thebibliography}

% 附录
\newpage
\section*{附\quad 录A\quad ××××××}
\addcontentsline{toc}{section}{附录A ××××××}
附录内容......

% 致谢
\newpage
\section*{致\quad 谢}
\addcontentsline{toc}{section}{致谢}
感谢我的指导教师×××教授......

% 个人简历
\newpage
\section*{个人简历及在学期间科研成果}
\subsection*{个人简历:}
XXX,男,XXXX年X月X日生......

\subsection*{已发表的学术论文与研究成果:}
[1] 作者. 论文题目. 刊物名称, 时间, 期数.

\end{document}
